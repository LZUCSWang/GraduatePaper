% 这是一个基于XeLaTeX的beamer幻灯片
\documentclass[11pt, aspectratio=169]{beamer}  % 将字体大小改为11pt

% 中文支持
\usepackage[UTF8]{ctex}
\usepackage{svg} % SVG图像支持

\usepackage{graphicx}
\usepackage{amsmath}
\usepackage{booktabs}
\usepackage{xcolor}
\usepackage{tikz}
\usetikzlibrary{shapes.geometric, arrows, positioning, calc} % ADDED THIS LINE for flowchart
\usepackage{hyperref}
\usepackage{listings}
\usepackage{multirow}  % 支持表格行合并
\usepackage{booktabs}
\usepackage{subfig}
\usepackage{algorithm}
\usepackage{algorithmicx}
\usepackage{algpseudocode}
% 添加图片序号设置
\setbeamertemplate{caption}[numbered]

% 修改代码显示的字体大小
\lstset{
    language=Python,
    breaklines=true,
    showstringspaces=false,
    tabsize=4,
    numbers=none,  % 移除行号以节省空间
    frame=none     % 秽除边框以减少视觉干扰
}

% 调整数学公式字体大小
\DeclareMathSizes{11}{11}{9}{7}  % 调整为对应11pt的数学字体大小

% 减小标题和段落间距
\setbeamertemplate{items}[circle]  % 使用更紧凑的项目符号
\setlength{\parskip}{0.5em}       % 减小段落间距

% 调整页边距以容纳更多内容
\setbeamersize{
    text margin left=0.5cm,
    text margin right=0.5cm
}

\usetheme{Madrid}
\usecolortheme{default}
\usepackage{NJUPT} % 加载NJUPT样式文件

% --- BEGIN HACK for total frame number ---
\makeatletter
\let\beamer@orig@inserttotalframenumber\inserttotalframenumber
\renewcommand{\inserttotalframenumber}{%
  \ifnum\c@framenumber<3\relax % 针对幻灯片序号小于3(即第1和第2张幻灯片)
    18 % 将总页数强制设为18
  \else
    \beamer@orig@inserttotalframenumber % 其他幻灯片使用原始的总页数命令
  \fi
}
\makeatother
% --- END HACK ---

\title{面向草原生态修复的多无人机协同\\路径规划与面积分配方法研究}
\subtitle{近期工作总结}
\author{王贤义}
\institute{兰州大学信息科学与工程学院}
\date{2025年5月19日}

% 在标题页后添加目录
\AtBeginDocument{
  \frame{\titlepage}
  \begin{frame}{目录}
    \tableofcontents[sectionstyle=show, subsectionstyle=hide]
  \end{frame}
}

\begin{document}

\section{引言与背景}
\begin{frame}
	\centering % 水平居中
	{\Huge \sffamily\bfseries\textcolor{njupt}{引言与背景}}
	\par % 换行
	\vspace{0.5cm} % 调整间距
	{\large \itshape{本部分将介绍研究的背景、意义及主要内容。}} % 修改后的引言
	% \vfill % 移除垂直居中
\end{frame}

\subsection{研究背景、意义与主要研究内容}
\begin{frame}
	\frametitle{研究背景、意义与主要研究内容}
	\begin{columns}[T]
		\begin{column}{0.5\textwidth} % Left column for 研究背景与意义
			% \footnotesize % Remove footnotesize
			\small % Change to small
			\textbf{研究背景与意义}
			\begin{itemize}
				\item \textbf{草原生态系统:}全球重要(占陆地26\%--40\%),具防风固沙、涵养水源、调节气候、维系生物多样性等关键功能。
				\item \textbf{草原退化挑战:}面临植被减少、土壤沙化、水土流失等问题,受人类活动与气候变化双重压力。
				\item \textbf{传统修复局限:}人工与机械修复效率低,大范围复杂地形下成本高、难度大。
				\item \textbf{无人机新机遇:}凭借机动、远程、低成本优势,多无人机协同为草原修复提供新途径。
			\end{itemize}
		\end{column}
		\begin{column}{0.5\textwidth} % Right column for 主要研究内容
			% \footnotesize % Remove footnotesize
			\small % Change to small
			\textbf{主要研究内容}
			\begin{itemize}
				\item 构建多无人机协同修复数学模型(修复面积最大化)。
				\item 研发基于深度强化学习的路径规划与面积分配算法。
				\item 设计多无人机协同调度架构及决策执行流程。
				\item 仿真实验验证与性能对比。
				\item 技术核心:Transformer编码器 + 指针网络解码器,Actor--Critic训练。
			\end{itemize}
		\end{column}
	\end{columns}
\end{frame}

\section{核心方法与模型}
\begin{frame}
	\centering % 水平居中
	{\Huge \sffamily\bfseries\textcolor{njupt}{核心方法与模型}}
	\par % 换行
	\vspace{0.5cm} % 调整间距
	{\large \itshape{本章将详细阐述问题建模、能量消耗模型及所提出的深度神经网络模型。}} % 修改后的引言
	% \vfill % 移除垂直居中
\end{frame}

\subsection{草原修复问题建模}
\begin{frame}
	\frametitle{草原修复问题建模}
	\begin{columns}
		\begin{column}{0.4\textwidth}
			\small % Add \small here
			\begin{itemize}
				\item 将草原建模为无向图 $G = (V, E)$
				      \begin{itemize}
					      \item $V = \{v_0, v_1, ..., v_N \}$ 表示待修复区域
					      \item $v_0$ 为地面信息融合中心
					      \item 每个区域具有位置、退化度、面积等属性
				      \end{itemize}
				\item 无人机特性
				      \begin{itemize}
					      \item 初始能量:$E_{max}$
					      \item 携带草种重量:$Q$
					      \item 退化程度范围:[0.3, 0.8]
				      \end{itemize}
			\end{itemize}
		\end{column}
		\begin{column}{0.6\textwidth}
			\centering
			\includesvg[width=1\textwidth]{figures/多无人机修复退化区域实例.svg}
			\captionof{figure}{多无人机协同草原退化区域示意图}
		\end{column}
	\end{columns}
\end{frame}

\subsection{无人机能量消耗模型}
\begin{frame}
	\frametitle{无人机能量消耗模型}
	\begin{columns}
		\begin{column}{0.55\textwidth}
			无人机在飞行过程中功率消耗:
			\begin{equation*}
				P(\bar{q}_{ij}) = (M + \bar{q}_{ij})^{\frac{3}{2}}\sqrt{\frac{g^3}{2 \rho \varsigma h}}
			\end{equation*}
			\textbf{三种能量消耗:}
			\small
			\begin{align*}
				\small
				E_f    & = \sum^N_{i=0}\sum^N_{j \neq i} e^f_{ij} d_{ij} x_{ij} \quad \text{(飞行能耗)} \\
				E_s    & = \sum^N_{i = 1}\sum^N_{j \neq i}\sigma_i e_i x_{ij} \quad \text{(播种能耗)}   \\
				E_{ap} & = e_{ap}\sum^N_{i=1}\sum^N_{j \neq i} x_{ij}\sigma_i \quad \text{(信息采集能耗)}
			\end{align*}
		\end{column}
		\begin{column}{0.45\textwidth}
			\begin{itemize}
				\item $M = W + m$:无人机自重
				\item $\bar{q}_{ij}$:当前草种重量
				\item $\sigma_i$:修复的单位圆数量
				\item $x_{ij}$:路径选择变量
				\item $e_i = \eta q_i$:单位面积播种能耗
				\item $q_i = (1 + l_i) \gamma$:单位面积草种重量
				\item $e_{ap}$:信息采集能耗系数
			\end{itemize}
		\end{column}
	\end{columns}
\end{frame}

\subsection{优化目标与约束条件}
\begin{frame}
	\frametitle{优化目标与约束条件}
	\begin{columns}
		\begin{column}{0.5\textwidth}
			\textbf{优化目标:加权修复面积最大化}
			\begin{equation*}
				\max_{x_{ij}, \sigma_i} \sum_{i=1}^{N} (l_i + 0.7) \cdot \sigma_i
			\end{equation*}

			\textbf{核心约束条件:}
			\begin{itemize}
				\item 能量约束:$E_s + E_{ap} + E_f \leq E_{max}$
				\item 载荷约束:草种重量平衡与不超限
				\item 路径约束:每个区域最多访问一次
				\item 面积约束:$1 \leq \sigma_i \leq c_i$
			\end{itemize}
		\end{column}
		\begin{column}{0.5\textwidth}
			\textbf{马尔可夫决策过程建模:}
			\begin{itemize}
				\item 状态 $s(<i)$:部分修复解
				\item 动作 $\pi_i=s(i)$:下一步修复决策
				\item 策略函数:$p(\pi|s)=\prod_{i=1}^n p(s(i)|s(<i))$
				\item 奖励函数:
				      \begin{equation*}
					      R(\pi|V)=\alpha_{p}*Pel+\alpha_{r}*\sum_{i=1}^{n}(l_i+0.7)*a_{i}
				      \end{equation*}
			\end{itemize}
		\end{column}
	\end{columns}
	\vspace{0.3cm}
	\textbf{模型特点:}多变量组合优化问题、NP-hard难度、非线性约束
\end{frame}

\subsection{深度神经网络模型构建}
\begin{frame}
	\frametitle{深度神经网络模型构建}
	\begin{columns}
		\begin{column}{0.5\textwidth}
			\footnotesize
			\textbf{Actor-Critic 数据流程说明:}

			模型采用 Actor-Critic 架构:
			\begin{itemize}
				\item \textbf{输入与编码:} 当前环境状态(无人机位置、电量,区域状况等)输入编码器 (Transformer),提取关键特征。
				\item \textbf{Actor (策略网络):} 解码器 (指针网络) 根据特征和状态评估,决定下一步动作(如目标区域、修复面积)。
				\item \textbf{Critic (价值网络):} 评估当前状态的潜在价值(预期总奖励),指导 Actor 优化决策。
				\item \textbf{训练与优化:} Actor 和 Critic 协同训练。Critic 通过比较预期与实际奖励更新评估;Actor 根据 Critic 评估和反馈调整策略,以获更高奖励。
			\end{itemize}
		\end{column}
		\begin{column}{0.5\textwidth}
			\centering
			\resizebox{\textwidth}{!}{%
				\begin{tikzpicture}[node distance=1.2cm and 0.8cm, >=stealth', every node/.style={draw, rectangle, rounded corners, minimum height=0.8cm, text width=2.5cm, align=center, font=\footnotesize}]
					% 定义节点
					\node (input) [ellipse, fill=blue!20] {输入 (状态 $s_t$)};
					\node (encoder) [below=of input, fill=green!20] {编码器 (Transformer)};
					\node (decoder) [below=of encoder, fill=green!20] {解码器 (指针网络)};
					\node (action) [below=of decoder, ellipse, fill=blue!20] {输出 (动作 $a_t$)};
					\node (critic) [right=of encoder, xshift=1.5cm, fill=orange!20] {Critic网络};
					\node (value) [below=of critic, ellipse, fill=blue!20] {状态值 $V(s_t)$};

					% 定义箭头
					\draw[->] (input) -- (encoder);
					\draw[->] (encoder) -- (decoder);
					\draw[->] (decoder) -- (action);
					\draw[->] (encoder) -- (critic);
					\draw[->] (critic) -- (value);
					% Simplified dashed lines - direct connection for value to decoder
					\draw[->, dashed] (value.west) -- (decoder.east); % Direct connection from value's left to decoder's right
					% \draw[->, dashed] (action.east) -- ($(action.east -| critic.east) + (0.4cm,0)$) coordinate (intermediate_for_action_critic) -- (intermediate_for_action_critic |- critic.east) -- (critic.east); % Previous attempt
					\draw[->, dashed] (action.east) -- ($(action.east -| critic.east) + (1.0cm,0)$) coordinate (intermediate_for_action_critic) -- (intermediate_for_action_critic |- critic.east) -- (critic.east);
				\end{tikzpicture}%
			}
			\captionof{figure}{Actor-Critic 数据流程}
		\end{column}
	\end{columns}
\end{frame}

\subsection{多无人机协同调度算法}
\begin{frame}
	\frametitle{多无人机协同调度算法}
	\begin{columns}[T]
		\begin{column}{0.5\textwidth}
			\footnotesize
			\textbf{多无人机协同调度算法流程:}

			采用分布式决策与中心协调结合,优化多无人机修复效率:
			\begin{itemize}
				\item \textbf{初始化:} 分配初始区域,确定决策优先级。
				\item \textbf{两阶段路径规划:}
				      \begin{enumerate}
					      \item 无人机初步规划并上报。
					      \item 中心整合信息,下发优化地图。
					      \item 无人机二次规划选优。
				      \end{enumerate}
				\item \textbf{协同决策与执行:} 高优先级无人机决策并执行任务。
				\item \textbf{动态调整:} 实时更新地图、状态、优先级,直至任务完成。
			\end{itemize}
			旨在提高总修复面积并有效利用无人机资源。
		\end{column}
		\begin{column}{0.5\textwidth}
			\centering
			\resizebox{0.8\columnwidth}{!}{%
				\begin{tikzpicture}[node distance=0.6cm and 1.5cm, >=stealth', % Adjusted node distance for vertical and horizontal spread
						main_node/.style={draw, rectangle, rounded corners, minimum height=0.8cm, text width=3.2cm, align=center, font=\footnotesize, fill=green!20},
						cond_node/.style={draw, diamond, aspect=1.5, minimum height=0.8cm, text width=2.2cm, align=center, font=\footnotesize, fill=orange!20},
						io_node/.style={draw, ellipse, minimum height=0.7cm, text width=1.8cm, align=center, font=\footnotesize, fill=blue!20}
					]
					% Nodes
					\node (start) [io_node] {开始};
					\node (init) [main_node, below=of start, yshift=-0.1cm] {初始化:\\分配区域,\\定优先级};
					\node (loop_cond) [cond_node, below=of init, yshift=-0.1cm] {所有任务\\完成?};

					\node (plan1) [main_node, below=of loop_cond, yshift=-0.2cm] {阶段1: 无人机\\初步规划与上报};
					\node (center_update) [main_node, right=of plan1, xshift=1.5cm] {阶段2: 中心整合\\与下发优化地图};
					\node (plan2) [main_node, below=of center_update, yshift=-0.2cm] {阶段3: 无人机\\二次规划选优};
					\node (collab_exec) [main_node, left=of plan2, xshift=-1.5cm] {协同决策与执行:\\高优先级无人机\\决策并执行}; % Moved left of plan2
					\node (dynamic_adjust) [main_node, below=of collab_exec, yshift=-0.2cm] {动态调整:\\更新地图、状态、\\优先级}; % Moved below new collab_exec

					\node (end) [io_node, right=of loop_cond, xshift=2.5cm] {结束};

					% Connections
					\draw[->] (start) -- (init);
					\draw[->] (init) -- (loop_cond);

					\draw[->] (loop_cond.south) -- node[left, pos=0.3, font=\tiny] {否} (plan1);
					\draw[->] (plan1) -- (center_update);
					\draw[->] (center_update) -- (plan2);
					\draw[->] (plan2) -- (collab_exec); % Arrow from plan2 to new collab_exec position
					\draw[->] (collab_exec) -- (dynamic_adjust); % Arrow from new collab_exec to new dynamic_adjust position

					% Loop back from dynamic_adjust to loop_cond
					% \draw[->] (dynamic_adjust.north) |- (loop_cond.west); % Previous polyline loop back
					\draw[->] (dynamic_adjust.south) -- ++(0,-0.4cm) -- ++(-2.5cm,0) |- (loop_cond.west);

					% "Yes" branch from loop_cond to end
					\draw[->] (loop_cond.east) -- node[above, pos=0.5, font=\tiny] {是} (end);
				\end{tikzpicture}%
			}
			\captionof{figure}{多无人机协同调度算法流程图} % Updated caption
		\end{column}
	\end{columns}
\end{frame}

\section{实验与结论}
\begin{frame}
	\centering % 水平居中
	{\Huge \sffamily\bfseries\textcolor{njupt}{实验与结论}}
	\par % 换行
	\vspace{0.5cm} % 调整间距
	{\large \itshape{本章将展示仿真实验结果并对研究工作进行总结与展望。}} % 修改后的引言
	% \vfill % 移除垂直居中
\end{frame}

\subsection{基于深度强化学习的求解方法}
\begin{frame}
	\frametitle{基于深度强化学习的求解方法}
	\begin{columns}
		\begin{column}{0.5\textwidth}
			\centering
			\includesvg[width=0.95\textwidth]{figures/actor_loss_curve.svg}
			\captionof{figure}{训练过程中修复面积变化 (Actor Loss)}
		\end{column}
		\begin{column}{0.5\textwidth}
			\centering
			\includesvg[width=0.95\textwidth]{figures/avg_repair_area_curve.svg}
			\captionof{figure}{训练过程中修复面积变化 (Avg Repair Area)}
		\end{column}
	\end{columns}
\end{frame}



\subsection{实验结果 - 算法性能对比}
\begin{frame}
	\frametitle{实验结果 - 算法性能对比\footnote{DRL: Deep Reinforcement Learning,本文提出的方法;CHAPBILM: Construction Heuristic Algorithm for Path Planning and Balanced Iterated Local Search for Mission,对比基准算法。}}
	\begin{table}
		\centering
		\caption{路径长度与修复面积对比(DRL 与 CHAPBILM)}
		\small % 使用更大的字体
		\setlength{\tabcolsep}{3pt}
		\begin{tabular}{ccc ccc ccc}
			\toprule
			\multirow{2}{*}{区域数} & \multirow{2}{*}{草原边长} & \multirow{2}{*}{ 无人机数} & \multicolumn{3}{c}{路径长度} & \multicolumn{3}{c}{修复面积}                                                        \\
			\cmidrule(lr){4-6}\cmidrule(lr){7-9}
			                     &                       &                        & DRL                      & CHAPBILM                 & Gap(\%)         & DRL    & CHAPBILM & Gap(\%)        \\
			\midrule
			\multirow{3}{*}{60}
			                     & 500                   & 4                      & 9396.35                  & 12648.21                 & \textbf{-25.68} & 267.00 & 194.00   & \textbf{37.63} \\
			                     & 600                   & 6                      & 26523.73                 & 27174.68                 & \textbf{-2.39}  & 294.00 & 206.00   & \textbf{42.72} \\
			                     & 700                   & 8                      & 37585.42                 & 36859.35                 & 1.97            & 238.00 & 254.00   & -6.30          \\
			\midrule
			\multirow{3}{*}{80}
			                     & 500                   & 4                      & 10221.88                 & 17217.03                 & \textbf{-40.60} & 370.00 & 315.00   & \textbf{17.46} \\
			                     & 600                   & 6                      & 29742.35                 & 28044.73                 & 6.05            & 406.00 & 282.00   & \textbf{44.00} \\
			                     & 700                   & 8                      & 44308.24                 & 42493.45                 & 4.28            & 355.00 & 311.00   & \textbf{14.14} \\
			\bottomrule
		\end{tabular}
		\label{tab:combined_comparison}
	\end{table}
\end{frame}
\subsection{实验结果 - 路径规划可视化}
\begin{frame}
	\frametitle{实验结果 - 路径规划可视化}
	\begin{figure} % 将columns和caption包裹在figure环境中
		\begin{columns}[T] % Align columns at the top
			\begin{column}{0.48\textwidth}
				\centering
				\textbf{CHAPBILM 方法}\par\vspace{1mm}
				\includesvg[width=0.3\linewidth]{figures/CHAPBILM/cvrp_60_500_4_multi_route_CHAPBILM.svg}\hfill
				\includesvg[width=0.3\linewidth]{figures/CHAPBILM/cvrp_60_500_6_multi_route_CHAPBILM.svg}\hfill
				\includesvg[width=0.3\linewidth]{figures/CHAPBILM/cvrp_60_500_8_multi_route_CHAPBILM.svg}\par\vspace{0.5mm}
				\includesvg[width=0.3\linewidth]{figures/CHAPBILM/cvrp_60_600_4_multi_route_CHAPBILM.svg}\hfill
				\includesvg[width=0.3\linewidth]{figures/CHAPBILM/cvrp_60_600_6_multi_route_CHAPBILM.svg}\hfill
				\includesvg[width=0.3\linewidth]{figures/CHAPBILM/cvrp_60_600_8_multi_route_CHAPBILM.svg}\par\vspace{0.5mm}
				\includesvg[width=0.3\linewidth]{figures/CHAPBILM/cvrp_60_700_4_multi_route_CHAPBILM.svg}\hfill
				\includesvg[width=0.3\linewidth]{figures/CHAPBILM/cvrp_60_700_6_multi_route_CHAPBILM.svg}\hfill
				\includesvg[width=0.3\linewidth]{figures/CHAPBILM/cvrp_60_700_8_multi_route_CHAPBILM.svg}
			\end{column}
			\hfill
			\begin{column}{0.48\textwidth}
				\centering
				\textbf{DRL 方法}\par\vspace{1mm}
				\includesvg[width=0.3\linewidth]{figures/DRL/cvrp_60_500_4_multi_route_DRL.svg}\hfill
				\includesvg[width=0.3\linewidth]{figures/DRL/cvrp_60_500_6_multi_route_DRL.svg}\hfill
				\includesvg[width=0.3\linewidth]{figures/DRL/cvrp_60_500_8_multi_route_DRL.svg}\par\vspace{0.5mm}
				\includesvg[width=0.3\linewidth]{figures/DRL/cvrp_60_600_4_multi_route_DRL.svg}\hfill
				\includesvg[width=0.3\linewidth]{figures/DRL/cvrp_60_600_6_multi_route_DRL.svg}\hfill
				\includesvg[width=0.3\linewidth]{figures/DRL/cvrp_60_600_8_multi_route_DRL.svg}\par\vspace{0.5mm}
				\includesvg[width=0.3\linewidth]{figures/DRL/cvrp_60_700_4_multi_route_DRL.svg}\hfill
				\includesvg[width=0.3\linewidth]{figures/DRL/cvrp_60_700_6_multi_route_DRL.svg}\hfill
				\includesvg[width=0.3\linewidth]{figures/DRL/cvrp_60_700_8_multi_route_DRL.svg}
			\end{column}
		\end{columns}
		\vspace{-5pt}
		\small
		\captionof{figure}{路径规划结果对比 60个待修复区域} % Moved \scriptsize inside
	\end{figure}
\end{frame}
\subsection{研究结论与展望}
\begin{frame}
	\frametitle{研究结论与展望}
	\begin{itemize}
		\item \textbf{研究结论:}
		      \begin{itemize}
			      \item 提出了面向草原生态修复的多无人机协同路径规划与面积分配方法,成功构建了数学模型并设计了高效算法。
			      \item 通过深度强化学习与Actor-Critic框架,显著提升了修复面积与路径规划的优化能力。
			      \item 实验结果表明,所提方法在多种场景下均优于传统算法,具有较好的适应性与鲁棒性。
		      \end{itemize}
		\item \textbf{研究展望:}
		      \begin{itemize}
			      \item 深入探索多智能体强化学习算法在大规模分布式任务中的应用,特别是值分解与策略梯度方法的结合。
			      \item 研究算法在动态环境下的自适应能力,提升对突发事件的响应速度与处理能力。
			      \item 拓展研究至其他类型的生态修复与资源管理问题,验证算法的通用性与有效性。
		      \end{itemize}
	\end{itemize}
\end{frame}
\section{未来工作计划}
\begin{frame}
	\centering
	{\Huge \sffamily\bfseries\textcolor{njupt}{未来工作计划}}
	\par
	\vspace{0.5cm}
	{\large \itshape{本部分将介绍后续的研究计划与展望。}}
\end{frame}

% UNIFIED FRAME FOR FUTURE WORK PLAN
\begin{frame}
    \frametitle{未来工作计划}
    \small % Use small for the paragraph
    \textbf{1. 完成当前模型仿真与分析:}
    为进一步验证和优化当前研究成果,将首先围绕以下核心参数完成预定的仿真实验。主要参数设置范围如下表所示:
    \vspace{0.5em} % Reduced space before the table
	\begin{table}[H] % Using [H] to suggest placing the table here if possible
        \centering
        \caption{仿真实验核心参数}
        \setlength{\tabcolsep}{10pt}
        \begin{tabular}{ccc}
            \toprule % Corrected from \\toprule
            参数  & 描述       & 值                               \\
            \midrule % Corrected from \\midrule
            $N$ & 待修复区数量   & $60, 80, 100, 120, 140, 160$    \\
            $U$ & 无人机数量    & $4, 6, 8$                       \\
            $L$ & 单无人机活动范围 & $500, \ldots, 1000$ \\ % Corrected ellipsis
            \bottomrule
        \end{tabular}
        \label{tbl_instance_setting_future_work}% Ensure label is unique
    \end{table}
    \vspace{1em}

    \textbf{2. 探索与对比多智能体强化学习 (MARL) 算法:}
    \begin{itemize}
        \item \textit{算法调研与实现:} 针对上述仿真实例,研究并实现主流MARL算法(如QMIX, VDN, MAPPO, MADDPG),重点关注大规模分布式协同任务。
        \item \textit{综合性能对比评估:} 将当前DRL方法与MARL算法在修复效果、资源消耗、计算效率、收敛性等方面进行全面对比。
        \item \textit{算法适应性与扩展性分析:} 评估不同算法在不同问题规模及动态环境下的表现与鲁棒性。
    \end{itemize}
\end{frame}



\begin{frame}
	\begin{center}
		\vspace{2cm}
		{\Huge \sffamily\bfseries\textcolor{njupt}{恳请各位老师批评指正!}} % 使用主题色 njupt
		\vspace{0.5cm}

		{\large \itshape\textcolor{gray}{Thanks}}
	\end{center}
\end{frame}

\end{document}
